%2multibyte Version: 5.50.0.2953 CodePage: 1252
%% This document created by Scientific Word (R) Version 2.0
%\usepackage{sw20elba}
%\input tcilatex


\documentclass[thmsa,notitlepage,11pt]{article}
%%%%%%%%%%%%%%%%%%%%%%%%%%%%%%%%%%%%%%%%%%%%%%%%%%%%%%%%%%%%%%%%%%%%%%%%%%%%%%%%%%%%%%%%%%%%%%%%%%%%%%%%%%%%%%%%%%%%%%%%%%%%%%%%%%%%%%%%%%%%%%%%%%%%%%%%%%%%%%%%%%%%%%%%%%%%%%%%%%%%%%%%%%%%%%%%%%%%%%%%%%%%%%%%%%%%%%%%%%%%%%%%%%%%%%%%%%%%%%%%%%%%%%%%%%%%
%TCIDATA{TCIstyle=Article/art1.lat,elba,article}

%TCIDATA{OutputFilter=LATEX.DLL}
%TCIDATA{Version=5.50.0.2953}
%TCIDATA{Codepage=1252}
%TCIDATA{<META NAME="SaveForMode" CONTENT="1">}
%TCIDATA{BibliographyScheme=Manual}
%TCIDATA{Created=Tue Nov 13 06:24:53 2001}
%TCIDATA{LastRevised=Wednesday, December 10, 2014 21:16:42}
%TCIDATA{<META NAME="GraphicsSave" CONTENT="32">}
%TCIDATA{Language=American English}

\input{tcilatex}
\begin{document}

\author{Richard Dennis\thanks{%
Email: richard.dennis@glasgow.ac.uk} \\
%EndAName
University of Glasgow}
\title{\textbf{Package Guide for SolveDSGE}\vspace{0.2in}}
\date{December 1, 2014}
\maketitle

\thispagestyle{empty}\newpage \setlength{\baselineskip}{18.95pt}%
\setcounter{page}{1}

\section{Introduction}

SolveDSGE is a Julia package for solving Dynamic Stochastic General
Equilibrium (DSGE) models. \ The package is aimed at macroeconomists
interested in first-order-accurate or second-order-accurate solutions to
their general equilibrium models. \ Although there is much that this package
does not do, SolveDSGE offers a broad array of solution methods that can be
applied provided the DSGE model can be expressed in one of several standard
dynamic representations.

\section{Installation}

To install SolveDSGE you will need to type the following into the Julia REPL

\bigskip

\textbf{Pkg.clone("git://github.com/RJDennis/SolveDSGE.git")}

\bigskip

\section{First-order-accurate solution methods}

SolveDSGE provides a range of solution methods for computing first-order
accurate solutions. \ Exploiting Julia's multiple dispatch all of these
solution methods are called via the single command \textbf{solve\_re}(). \
From this single command the particular solution method employed depends
principally on the model type that enters the \textbf{solve\_re()} function
call. \ Models are represented in various forms that are summarized by
types. \ The model types are

\begin{itemize}
\item \textbf{Blanchard\_Kahn\_Form}

\item \textbf{Klein\_Form}

\item \textbf{Sims\_Form}

\item \textbf{Binder\_Pesaran\_Form}
\end{itemize}

\subsection{Blanchard-Kahn Form}

The Blanchard-Kahn Form is given by%
\[
\left[ 
\begin{array}{c}
x_{t+1} \\ 
\text{E}_{t}y_{t+1}%
\end{array}%
\right] =A\left[ 
\begin{array}{c}
x_{t} \\ 
y_{t}%
\end{array}%
\right] +C\left[ \epsilon _{t+1}\right] , 
\]%
$\epsilon _{t}\sim i.i.d.[0,\Sigma ]$, where $x_{t}$ is an $n_{x}\times 1$
vector of predetermined variables and $y_{t}$ is an $n_{y}\times 1$ vector
of non-predetermined variables. \ To solve models of this form we first
create the \textbf{Blanchard\_Kahn\_Form} type for the model, then we use 
\textbf{solve\_re()} to solve the model. \ The relevant lines of code would
be something like

\bigskip

\textbf{cutoff = 1.0}

\textbf{model = Blanchard\_Kahn\_Form(nx,ny,a,c,sigma)}

\textbf{soln = solve\_re(model,cutoff)}

\bigskip

Here \textbf{soln} contains the solution, which is of the form%
\begin{eqnarray*}
x_{t+1} &=&Px_{t}+K\epsilon _{t+1}, \\
y_{t} &=&Fx_{t},
\end{eqnarray*}%
and information about the number of eigenvalues with modulus greater than
cutoff and whether the \textquotedblleft solution\textquotedblright\
returned is determinate, indeterminate, or explosive.

Models of the \textbf{Blanchard\_Kahn\_Form} type can also be solved using
an iterative method to solve a non-symmetric, continuous, algebraic, Riccati
equation. \ In this case the relevant lines of code might look like

\bigskip

\textbf{tol = 1e-10}

\textbf{cutoff = 1.0}

\textbf{model = Blanchard\_Kahn\_Form(nx,ny,a,c,sigma)}

\textbf{soln = solve\_re(model,cutoff,tol)}

\bigskip

For this iterative method, the variable \textbf{cutoff} is still used to
establish determinacy, but is not used to order eigenvalues.

\subsection{Klein Form}

The Klein Form for a model is given by%
\[
B\left[ 
\begin{array}{c}
x_{t+1} \\ 
\text{E}_{t}y_{t+1}%
\end{array}%
\right] =A\left[ 
\begin{array}{c}
x_{t} \\ 
y_{t}%
\end{array}%
\right] +C\left[ \epsilon _{t+1}\right] , 
\]%
$\epsilon _{t}\sim i.i.d.[0,\Sigma ]$, where $x_{t}$ is an $n_{x}\times 1$
vector of predetermined variables, $y_{t}$ is an $n_{y}\times 1$ vector of
non-predetermined variables, and $B$ need not have full rank. \ We can solve
models of this form using the code

\bigskip

\textbf{cutoff = 1.0}

\textbf{model = Klein\_Form(nx,ny,a,b,c,sigma)}

\textbf{soln = solve\_re(model,cutoff) }

\bigskip

The composite type \textbf{soln} contains the solution which is of the form%
\begin{eqnarray*}
x_{t+1} &=&Px_{t}+K\epsilon _{t+1}, \\
y_{t} &=&Fx_{t},
\end{eqnarray*}%
as well as information about the number of eigenvalues that have modulus
greater than cutoff and whether the \textquotedblleft
solution\textquotedblright\ returned is determinate, indeterminate, or
explosive.

\subsection{Sims Form}

An alternative model form is used by Sims (2000) and is given by%
\[
\Gamma _{0}z_{t}=\Gamma _{1}z_{t-1}+C+\Psi v_{t}+\Pi \eta _{t}, 
\]%
where $v_{t}$ is a shock process, possibly serially correlated, with
mean-zero innovations whose variance-covarince matrix is given by $\Sigma $.
\ To solve models that are in this form we would do something like the
following

\bigskip

\textbf{cutoff = 1.0}

\textbf{model = Sims\_Form(gamma0,gamma1,c,psi,pi,sigma)}

\textbf{soln = solve\_re(model,cutoff)}

\bigskip

Here the solution, summarized by \textbf{soln}, is of the form%
\[
z_{t}=G_{1}z_{t-1}+c+impact\times v_{t}+ywt\times \left[ I-fmat\times L^{-1}%
\right] ^{-1}\times fwt\times v_{t+1}. 
\]

\subsection{Binder-Pesaran Form}

A model is in \textquotedblleft structural form\textquotedblright\ if it is
written as%
\[
Az_{t}=A_{1}z_{t-1}+B\text{E}_{t}z_{t+1}+C\epsilon _{t}, 
\]%
where $\epsilon _{t}\sim i.i.d.[0,\Sigma ]$. \ We have two ways of solving
structural form models. \ The first recasts them in terms of the Klein form
and here the relevant code would look something like

\bigskip

\textbf{cutoff = 1.0}

\textbf{model = Binder\_Pesaran\_Form(a,a1,b,c,sigma)}

\textbf{soln = solve\_re(model,cutoff)}

\bigskip

The second method is iterative, implementing Binder and Pesaran's
\textquotedblleft brute force\textquotedblright\ method; here the code would
be something like

\bigskip

\textbf{tol = 1e-10}

\textbf{cutoff = 1.0}

\textbf{model = Binder\_Pesaran\_Form(a,a1,b,c,sigma)}

\textbf{soln = solve\_re(model,cutoff,tol)}

\bigskip

Regardless of which of the two methods is used, the solution, summarized in 
\textbf{soln}, has the form%
\[
z_{t}=Pz_{t-1}+K\epsilon _{t}. 
\]%
As earlier, \textbf{soln} is a composite type that in addition to the
solution itself also contains information about the number of eigenvalues
with modulus greater than cutoff and whether the solution returned is
determinate, indeterminate, or explosive.

\section{Second-order-accurate solution methods}

In addition to the first-order accurate solution methods documented above,
SolveDSGE also contains two methods form obtaining second-order-accurate
solutions to nonlinear DSGE models. \ As coded here, these two methods
employ the same model form but differ in how the solution is computed. \ To
employ either method the DSGE model is first expressed in the form%
\[
\text{E}_{t}G\left( x_{t},y_{t},x_{t+1},y_{t+1}\right) =0. 
\]%
\ With $x_{t}$ containing $n_{x}$ predetermined variables and $y_{t}$
containing $n_{y}$ non-predetermined variables, $G()$ is a vector containing 
$n_{x}+n_{y}$ functions. \ Bundling $x_{t}$ and $y_{t}$ into a new vector $%
z_{t}=\left[ 
\begin{array}{cc}
x_{t}^{^{\prime }} & y_{t}^{^{\prime }}%
\end{array}%
\right] ^{^{\prime }}$ and bundling $z_{t}$ and $z_{t+1}$ into a new vector $%
p_{t}=\left[ 
\begin{array}{cc}
z_{t}^{^{\prime }} & z_{t+1}^{^{\prime }}%
\end{array}%
\right] ^{^{\prime }}$ we get%
\[
\text{E}_{t}G\left( p_{t}\right) =0. 
\]%
We now approximate $G\left( p_{t}\right) $ around the deterministic steady
state, $\overline{p}$, using a second-order Taylor expansion giving%
\[
G\left( p_{t}\right) \simeq G_{p}\left( p_{t}-\overline{p}\right) +\left[
I\otimes \left( p_{t}-\overline{p}\right) \right] ^{^{\prime }}G_{pp}\left[
I\otimes \left( p_{t}-\overline{p}\right) \right] =0, 
\]%
where $G_{p}$ is a matrix of first-derivatives and $G_{pp}$ is a matrix of
stacked Hessians, one Hessian for each of the $n_{x}+n_{y}$ equations.

We now recognize that some elements of $x_{t}$ (usually the first $s$
elements) are shocks that have the form%
\[
s_{t+1}=\Lambda s_{t}+\eta \epsilon _{t+1}, 
\]%
where $\epsilon _{t}\sim i.i.d.[0,\Sigma ]$. \ The essential components
required for a second-order-accurate solution are now given by $n_{x}$, $%
n_{y}$, $G_{p}$, $G_{pp}$, $\eta $, and $\Sigma $.

The two model types that we consider for second-order-accurate solution
methods are

\begin{itemize}
\item \textbf{Gomme\_Klein\_Form}

\item \textbf{Lombardo\_Sutherland\_Form}
\end{itemize}

\subsection{Gomme-Klein Form}

To compute a second-order accurate solution using the Gomme and Klein (2011)
method we summarize the model in the form of the \textbf{Gomme\_Klein\_Form}
composite type. \ Once this model type is constructed the model can be
solved. \ The code to compute the solution would be something like

\bigskip

\textbf{cutoff = 1.0}

\textbf{model = Gomme\_Klein\_Form(nx,ny,Gp,Gpp,eta,sigma)}

\textbf{soln = solve\_re(model,cutoff)}

\bigskip

Here \textbf{soln} is a composite type that contains the solution, which is
of the form%
\begin{eqnarray*}
x_{t+1}-\overline{x} &=&\frac{1}{2}ssh+hx\left( x_{t}-\overline{x}\right) +%
\frac{1}{2}\left[ I\otimes \left( x_{t}-\overline{x}\right) \right]
^{^{\prime }}hxx\left[ I\otimes \left( x_{t}-\overline{x}\right) \right]
+\eta \epsilon _{t+1}, \\
y_{t}-\overline{y} &=&\frac{1}{2}ssg+gx\left( x_{t}-\overline{x}\right) +%
\frac{1}{2}\left[ I\otimes \left( x_{t}-\overline{x}\right) \right]
^{^{\prime }}gxx\left[ I\otimes \left( x_{t}-\overline{x}\right) \right] ,
\end{eqnarray*}%
where $hxx$ and $gxx$ are stacked matrices containing the second order
coefficients for each of the $n_{x}$ and $n_{y}$ equations, respectively. \ 
\textbf{soln} also contains information about the number of eigenvalues with
modulus greater than cutoff and the solution's determinacy properties, where
these properties are associated with the model first-order dynamics.

\subsection{Lombardo-Sutherland Form}

Implementing the Lombardo and Sutherland (2007) solution method is no
different than for Gomme and Klein (2010). \ The key difference is the form
in which the solution is presented. \ Ths code to implement the
Lombardo-Sutherland method would look something like

\bigskip

\textbf{cutoff = 1.0}

\textbf{model = Lombardo\_Sutherland\_Form(nx,ny,Gp,Gpp,eta,sigma)}

\textbf{soln = solve\_re(model,cutoff)}

\bigskip

where now the solution has the form%
\begin{eqnarray*}
x_{t+1}-\overline{x} &=&\frac{1}{2}ssh+hx\left( x_{t}-\overline{x}\right) +%
\frac{1}{2}hxx\times v_{t}+\eta \epsilon _{t+1}, \\
y_{t}-\overline{y} &=&\frac{1}{2}ssg+gx\left( x_{t}-\overline{x}\right) +%
\frac{1}{2}gxx\times v_{t}, \\
v_{t} &=&\Phi v_{t-1}+\Gamma vech\left( \epsilon _{t}\epsilon _{t}^{^{\prime
}}\right) +\Psi vec\left( x_{t}\epsilon _{t}^{^{\prime }}\right) ,
\end{eqnarray*}%
with $v_{t}$ given by%
\[
v_{t}=vech\left( x_{t}x_{t}^{\prime }\right) . 
\]

The solution form produced by the Lombardo-Sutherland method can be
converted into that produced by the Gomme-Klein method by using the \textbf{%
convert\_second\_order} function as follows

\bigskip

\textbf{new\_soln = convert\_second\_order(soln)}

\bigskip

Where \textbf{soln} is of type \textbf{Lombardo\_Sutherland\_Soln}, \textbf{%
new\_soln} is of type \textbf{Gomme\_Klein\_Soln}.

\setlength{\baselineskip}{10pt}

\begin{thebibliography}{9}
\bibitem{} Binder, M., and H. Pesaran, (1995), \textquotedblleft
Multivariate Rational Expectations Models and Macroeconomic Modeling: A
Review and Some New Results,\textquotedblright\ in: Pesaran, H., M. Wickens,
(Eds.), \textit{Handbook of Applied Econometrics}, Basil Blackwell, Oxford,
pp.139--187.

\bibitem{} Blanchard, O., and C. Kahn, (1980), \textquotedblleft The
Solution to Linear Difference Models under Rational
Expectations,\textquotedblright\ \textit{Econometrica}, 48, pp.1305--1311.

\bibitem{} Gomme, P., and P. Klein, (2011), \textquotedblleft Second-Order
Approximation of Dynamic Models Without the Use of
Tensors,\textquotedblright\ \textit{Journal of Economic Dynamics and Control}%
, 35, pp.604--615.

\bibitem{} Klein, P., (2000), \textquotedblleft Using the Generalized Schur
Form to Solve a Multivariate Linear Rational Expectations
Model,\textquotedblright\ \textit{Journal of Economic Dynamics and Control},
24, pp.1405--1423.

\bibitem{} Lombardo, G., and A. Sutherland, (2007), \textquotedblleft
Computing Second-Order-Accurate Solutions for Rational Expectations models
Using Linear Solution Methods,\textquotedblright\ \textit{Journal of
Economic Dynamics and Control}, 31, pp.515--530.

\bibitem{} Sims, C., (2001), \textquotedblleft Solving Linear Rational
Expectations Models,\textquotedblright\ \textit{Computational Economics},
20, pp.1--20.
\end{thebibliography}

\end{document}
