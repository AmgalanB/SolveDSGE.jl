
\documentclass[notitlepage,11pt]{article}
%%%%%%%%%%%%%%%%%%%%%%%%%%%%%%%%%%%%%%%%%%%%%%%%%%%%%%%%%%%%%%%%%%%%%%%%%%%%%%%%%%%%%%%%%%%%%%%%%%%%%%%%%%%%%%%%%%%%%%%%%%%%%%%%%%%%%%%%%%%%%%%%%%%%%%%%%%%%%%%%%%%%%%%%%%%%%%%%%%%%%%%%%%%%%%%%%%%%%%%%%%%%%%%%%%%%%%%%%%%%%%%%%%%%%%%%%%%%%%%%%%%%%%%%%%%%
\usepackage{amsfonts}
\usepackage{amsmath}
\usepackage{geometry}
\usepackage{graphicx}
\usepackage{amssymb}
\usepackage{makeidx}
\usepackage{multicol}

\setcounter{MaxMatrixCols}{10}
%TCIDATA{OutputFilter=Latex.dll}
%TCIDATA{Version=5.50.0.2960}
%TCIDATA{<META NAME="SaveForMode" CONTENT="1">}
%TCIDATA{BibliographyScheme=Manual}
%TCIDATA{LastRevised=Wednesday, May 06, 2020 22:07:37}
%TCIDATA{<META NAME="GraphicsSave" CONTENT="32">}
%TCIDATA{Language=American English}

\geometry{left=1in,right=1in,top=1.25in,bottom=1in}
\input{tcilatex}
\begin{document}

\author{Richard Dennis\thanks{%
Address for Correspondence: Adam Smith Business School, University of
Glasgow, Main Building, University Avenue, Glasgow G12 8QQ; email:
richard.dennis@glasgow.ac.uk.} \\
%EndAName
University of Glasgow and CAMA}
\title{SolveDSGE v0.3.7 --- A User Guide}
\date{April 2020}
\maketitle

\begin{abstract}
SolveDSGE is a Julia package for solving nonlinear Dynamic Stochastic
General Equilibrium models. \ A varietry of solution methods are available,
and they are interchangable so that one solution can be used subsequently as
an initialization to obtain a more accurate solution.\noindent\ \ The
package can compute one- second- and third-order perturbation solutions and
Chebyshev-based, Smolyak-based and piecewise linear-based projection
solutions. \ Once a model has been solved, the package can used used to
simulate data and/or compute impulse response functions.

\vspace{0.12in}\noindent {JEL Classification: E3, E4, E5.}
\end{abstract}

\thispagestyle{empty}\newpage \setlength{\baselineskip}{18.95pt}%
\setcounter{page}{1}

\section{Introduction}

SolveDSGE is a framework for solving and analyzing Dynamic Stochastic
General Equilibrium (DSGE) models that is implemented in the programming
language Julia. \ SolveDSGE will solve nonlinear DSGE models using
perturbation methods, producing solutions that are accurate to first,
second, and third order, but this is not its focus. \ The package's focus is
on applying projection methods to obtain solutions that are globally
accurate.

Obtaining globally accurate solutions to nonlinear DSGE models is
notoriously difficult. \ Solutions are invariably slow to obtain and
model-specific characteristics are often exploited to speed up the solution
process. \ SolveDSGE does not exploit model-specific characteristics in
order to solve a model. \ Instead, SolveDSGE applies the same general
solution strategy to all models. \ Nonetheless, making use of Julia's speed,
SolveDSGE allows models to be solved \textquotedblleft relatively
quickly\textquotedblright , and it provides users with an easy, unified, way
of organizing and expressing their model. \ At the users request, globally
accurate\ solutions can be obtained using Chebyshev polynomials, Smolyak
polynomials, or piecewise linear approximations, with the solution obtained
from one approximation scheme able to be used as an initialization for the
others, allowing greater speed and accuracy to be obtained via a form of
homotophy.

To use SolveDSGE to solve a model, two files must be supplied. \ The first
file (the model file) summarizes the model to be solved. \ The second file
(the solution file) reads the model file, solves the model, and performs any
post-solution analysis.

Quite a lot of time and effort has gone into writing SolveDSGE, together
with the underlying modules: ChebyshevApprox, SmolyakApprox, and
PiecewiseLinearApprox, but it is far from perfect. \ SolveDSGE may not be
able to solve your model, or it may not obtain a solution quickly enough to
be useful to you. \ You are welcome to suggest improvements to fix bugs or
add functionality. \ At the same time, I am hopeful that you will find the
package useful for your research. \ If it is, then please cite this User
Guide and add an acknowledgement of SolveDSGE to your paper/report.

\section{What types of models can be solved?}

SolveDSGE is designed to solve models that can be written in the following
standard form:%
\begin{equation}
E_{t}\left[ \mathbf{f}\left( \mathbf{x}_{t},\mathbf{y}_{t},\mathbf{x}_{t+1},%
\mathbf{y}_{t+1},\mathbf{\varepsilon }_{t+1}\right) \right] =\mathbf{0},
\end{equation}%
where $\mathbf{x}_{t}$ is a vector of state variables, $\mathbf{y}_{t}$ is a
vector of jump variables, and $\mathbf{\varepsilon }_{t+1}$ is a vector of
shocks. \ The first-order conditions and constraints for the DSGE model are
specified equation-by-equation. \ The shocks, state variables, and jump
variables are defined and then SolveDSGE takes the model and expresses it in
the form of equation (1) in preparation for solution. \ Equation (1) covers
a wide set of models, but obviously not all models. \ In principle SolveDSGE
can handle standard business cycle models of the real and new Keynesian
varieties, and it can handle models with volatility shocks, but it cannot
handle heterogeneous agents models, nor models with generalized Euler
equation like those that emerge from discretionary policy problems or from
models with quasi-geometric discounting. \ Allowing for models that contain
generalized Euler equations is a topic for future work. \ Another set of
models that are not fully accommodated are those where the shocks are
contemporaneously correlated. \ Such models can be solved via the
perturbation methods, but not via the projection methods (due to the
techniques used for quadrature).

\section{The model file}

SolveDSGE requires that the model that is to be solved be stored in a model
file. \ The model file is simply a text file so there is nothing
particularly special about it. \ Every model file must contain the following
five information categories: \textquotedblleft states:\textquotedblright ,
\textquotedblleft jumps:\textquotedblright , \textquotedblleft
shocks:\textquotedblright , \textquotedblleft parameters:\textquotedblright
, and \textquotedblleft equations:\textquotedblright ; each category name
must end with a colon. \ Each category will begin with its name, such as
\textquotedblleft states:\textquotedblright\ and conclude with an
\textquotedblleft end\textquotedblright . \ The model file can present these
five categories in any order.

The information in each category can be presented with one element per line,
or with multiple elements on each line with each element separated by either
a comma or a semi-colon. \ So if the jump variables in the model are labor,
consumption, and output, then this could be presented in a variety of ways,
such as:

\bigskip

\textit{jumps:}

\textit{labor}

\textit{consumption}

\textit{output}

\textit{end}

\bigskip

or:

\textit{jumps:}

\textit{labor, consumption, output}

\textit{end}

\bigskip

or:

\textit{jumps:}

\textit{labor; consumption, output}

\textit{end}

\bigskip

The first lag of a variable is denoted with a -1, so the lag of consumption
is consumption(-1). \ Similarly the first lead of a variable is denoted with
a +1, so the lead of consumption is denoted consumption(+1). \ The first lag
of any model variable is automatically included as a state variable, second
and higher lags should be given a name, defined by an equation, and included
as state variables explicitly. \ The package may allow higher lags to
processed automatically at a later stage.

Shocks in the model refers to the innovations to the shock processes, so if
the shock process is given by%
\begin{equation*}
tech(+1)=rho\ast tech+sd\ast epsilon,
\end{equation*}%
then \textquotedblleft $tech$\textquotedblright\ will be a state variable,
\textquotedblleft $epsilon$\textquotedblright\ will be a shock, and
\textquotedblleft $rho$\textquotedblright\ and \textquotedblleft $sd$%
\textquotedblright\ will be parameters. \ If the model is deterministic,
then it will contain no shocks.

Every element in the parameters category and the equations category must
contain an \textquotedblleft $=$\textquotedblright\ sign, such as
\textquotedblleft $alpha=0.33$\textquotedblright\ in the case of the
parameters category and \textquotedblleft $output=exp(tech)\ast capital%
\symbol{94}alpha\ast labor\symbol{94}(1.0-alpha)$\textquotedblright\ in the
case of the equations category.

\subsection{Example}

The following is an example of a model file for the stochastic growth model:

\bigskip

\textit{states:}

\textit{cap, tech}

\textit{end}

\textit{\bigskip }

\textit{jumps:}

\textit{cons}

\textit{end}

\textit{\bigskip }

\textit{shocks:}

\textit{epsilon}

\textit{end}

\textit{\bigskip }

\textit{parameters:}

\textit{betta = 0.99}

\textit{sigma = 1.1}

\textit{delta = 0.025}

\textit{alpha = 0.30}

\textit{rho = 0.8}

\textit{sd = 0.01}

\textit{end}

\textit{\bigskip }

\textit{equations:}

\textit{cap(+1) = (1.0 - delta)*cap + exp(tech)*cap\symbol{94}alpha - cons}

\textit{cons\symbol{94}(-sigma) = betta*cons(+1)\symbol{94}(-sigma)*(1.0 -
delta + alpha*exp(tech(+1))*cap(+1)\symbol{94}(alpha - 1.0))}

\textit{tech(+1) = rho*tech + sd*epsilon}

\textit{end}

\section{Solving a model}

Solving a model is straightforward; it consists of the following steps:

\begin{enumerate}
\item Read and process the model file. \ During the processing the order of
variables in the system may be changed, typically the changes are to place
the shocks at the top of the system. \ After processing is complete you will
be told what the variable-order is.

\item Solve for the model's steady state.

\item Specify a SolutionScheme. \ A SolutionScheme specifies the solution
method along with any parameters needed to implement that solution method.

\item Solve the model according to the chosen SolutionScheme.
\end{enumerate}

\subsection{Reading the model and solving for its steady state}

To read and process a model file we simply supply the path/filename to the
get\_model() function, for example

\bigskip

\textit{dsge = get\_model("c:/desktop/model.txt")}

\bigskip

We can then solve for the model's steady state as follows

\bigskip

\textit{ss = compute\_steady\_state(dsge,tol,maxiters)}

\bigskip

where \textit{dsge} is the model whose steady state is to be computed, 
\textit{tol} is a convergence tolerance, and \textit{maxiters} is an integer
specifying the maximum number of iterations before the function exits.

\subsection{Specifying a SolutionScheme}

To solve a model a SolutionScheme must be supplied. \ A SolutionScheme
specifies the solution method and the parameters upon which this solution
method relies. \ The solution methods in SolveDSGE are either perturbation
methods or projection methods. \ Accordingly, the SolutionSchemes can be
divided into PerturbationSchemes or ProjectionSchemes. \ We present each in
turn.

\subsubsection{PerturbationSchemes}

To solve a model using a perturbation method requires and
PerturbationScheme. \ Regardless of the model or the order of the
perturbation, a PerturbationScheme is a structure with three fields: the
point about which to perturb the model (the steady state), a cutoff
parameter that separates unstable from stable eigenvalues (eigenvalues whose
modulus is greater than cutoff will be placed in the model's unstable
block), and the order of the perturbation. \ For a first-order perturbation,
a typical PerturbationScheme might be the following

\bigskip

\textit{N = PerturbationScheme(ss,cutoff,"first")}

\bigskip

while those for second and third order perturbations might be

\bigskip

\textit{NN = PerturbationScheme(ss,cutoff,"second")}

\bigskip

and

\bigskip

\textit{NNN = PerturbationScheme(ss,cutoff,"third")}

\bigskip

The method used to compute a first-order perturbation follows Klein (2000),
that for a second-order perturbation follows Gomme and Klein (2011), while
that for a third-order perturbation follows Binning (2013) with a refinement
from Levintal (2017). \ At this point, perturbation solutions higher than
third order are not supported.

\subsubsection{ProjectionSchemes}

ProjectionSchemes are either ChebyshevSchemes, SmolyakSchemes, or
PiecewiseLinearSchemes, and for each of these there is a stochastic (for
stochastic models) and a deterministic (for deterministic models) version. \
The SolutionScheme for the deterministic case is a special case of the
stochastic one, so we focus on the stochastic case in what follows.

\paragraph{ChebyshevSchemes}

Solutions based on Chebyshev polynomials rely on and make use of all of the
functionality of the module ChebyshevApprox. \ This means that an arbitrary
number of state variables can be accommodated (if you have enough time!) and
both tenser-product and complete polynomials can be used. \ A stochastic
ChebyshevScheme requires the following arguments:

\begin{itemize}
\item initial\_guess --- This will usually be a vector containing the
model's steady state. \ It is used as the initial guess at the solution for
the case where an initializing solution is not provided (see the section on
model solution below).

\item node\_generator --- This is the name of the function used to generate
the nodes for the Chebyshev polynomial. \ Possible options include:
chebyshev\_nodes and chebyshev\_extrema.

\item node\_number --- This gives the number of nodes to be used for each
state variable. \ If there is only one state variables then node\_number
will be an integer. \ When there are two of more state variables it will be
a vector of integers.

\item num\_quad\_nodes --- This is an integer specifying the number of
quadrature points used to compute expectations.

\item order --- This defines the order of the Chebyshev polynomial to be
used in the approximating functions. \ For a complete polynomial order will
be an integer; for a tenser-product polynomial order will be a vector of
integers.

\item domain --- This contains the domain for the state variables over which
the solution is obtained. \ Domain will be a $2-$element vector in the
one-state-variable case and a $2\times n$ array in the $n$-state-variable
case, with the first row of the array containing the upper values of the
domain and the second row containing the lower values of the domain. \ If an
initializing solution is provided, then the domain associated with that
initializing solution can be used by setting domain to an empty array,
Float64[].

\item tol\_fix\_point\_solver --- This specifies the tolerance to be used in
the inner loop to determine convergence at each solution node.

\item tol\_variables --- This specifies the tolerance to be used in the
outer loop to determine convergence of the overall solution.

\item maxiters --- This is an integer specifying the maximum number of
outer-loop iterations before the solution exits.
\end{itemize}

\bigskip

An example of a stochastic ChebyshevScheme is

\bigskip

\textit{C = ChebyshevSchemeStoch(ss,chebyshev\_nodes,[21,21], 9, 4,[0.1
30.0; -0.1 20.0],1e-8,1e-6,1000)}

\bigskip

In the deterministic case the number of quadrature nodes is not needed, i.e.,

\bigskip

\textit{Cdet = ChebyshevSchemeDet(ss,chebyshev\_nodes,[21,21],4,[0.1 30.0;
-0.1 20.0],1e-8,1e-6,1000)}

\paragraph{SmolyakSchemes}

Underlying the Smolyak polynomial based solution is the module
SmolyakApprox. \ This module allows for both isotropic polynomials and
ansiotropic polynomials and several different methods for producing nodes. \
SolveDSGE exploits all of this functionality. \ A stochastic SmolyakScheme
requires the following arguments:

\begin{itemize}
\item initial\_guess --- This will usually be a vector containing the
model's steady state. \ It is used as the initial guess at the solution for
the case where an initializing solution is not provided (see the section on
model solution below).

\item node\_generator ---This is the name of the function used to generate
the nodes for the Smolyak polynomial. \ Possible options include:
chebyshev\_gauss\_lobatto and clenshaw\_curtis\_equidistant

\item num\_quad\_nodes --- This is an integer specifying the number of
quadrature points used to compute expectations.

\item layer --- This is an integer (isotropic case) or a vector of integers
(ansiotropic case) specifying the number of layers to be used in the
approximation.

\item domain --- This contains the domain for the state variables over which
the solution is obtained. \ Domain will be a $2-$element vector in the
one-state-variable case and a $2\times n$ array in the $n$-state-variable
case, with the first row of the array containing the upper values of the
domain and the second row containing the lower values of the domain. \ If an
initializing solution is provided, then the domain associated with that
initializing solution can be used by setting domain to an empty array,
Float64[].

\item tol\_fix\_point\_solver --- This specifies the tolerance to be used in
the inner loop to determine convergence at each solution node.

\item tol\_variables --- This specifies the tolerance to be used in the
outer loop to determine convergence of the overall solution.

\item maxiters --- This is an integer specifying the maximum number of
outer-loop iterations before the solution exits.
\end{itemize}

\bigskip

An example of a stochastic SmolyakScheme is

\bigskip

\textit{S = SmolyakSchemeStoch(ss,chebyshev\_gauss\_lobatto,9,3,[0.1 30.0;
-0.1 20.0],1e-8,1e-6,1000)}

\bigskip

In the deterministic case the number of quadrature nodes is not needed, i.e.,

\bigskip

\textit{Sdet = SmolyakSchemeDet(ss,chebyshev\_gauss\_lobatto,3,[0.1 30.0;
-0.1 20.0],1e-8,1e-6,1000)}

\paragraph{PiecewiseLinearSchemes}

To obtain piecewise linear solutions, SolveDSGE employs the module
PiecewiseLinearApprox, which allows approximations over an arbitrary number
of state variables. \ A stochastic PiecewiseLinearScheme requires the
following arguments:

\begin{itemize}
\item initial\_guess --- This will usually be a vector containing the
model's steady state. \ It is used as the initial guess at the solution for
the case where an initializing solution is not provided (see the section on
model solution below).

\item node\_number --- This gives the number of nodes to be used for each
state variable. \ If there is only one state variables then node\_number
will be an integer. \ When there are two of more state variables it will be
a vector of integers.

\item num\_quad\_nodes --- This is an integer specifying the number of
quadrature points used to compute expectations.

\item domain --- This contains the domain for the state variables over which
the solution is obtained. \ Domain will be a $2-$element vector in the
one-state-variable case and a $2\times n$ array in the $n$-state-variable
case, with the first row of the array containing the upper values of the
domain and the second row containing the lower values of the domain. \ If an
initializing solution is provided, then the domain associated with that
initializing solution can be used by setting domain to an empty array,
Float64[].

\item tol\_fix\_point\_solver --- This specifies the tolerance to be used in
the inner loop to determine convergence at each solution node.

\item tol\_variables --- This specifies the tolerance to be used in the
outer loop to determine convergence of the overall solution.

\item maxiters --- This is an integer specifying the maximum number of
outer-loop iterations before the solution exits.
\end{itemize}

\bigskip

An example of a stochastic PiecewiseLinearScheme is

\bigskip

\textit{P = PiecewiseLinearStoch(ss,[21,21],9,[0.1 30.0; -0.1
20.0],1e-8,1e-6,1000)}

\bigskip

In the deterministic case the number of quadrature nodes is not needed, i,e,,

\bigskip

\textit{Pdet = PiecewiseLinearDet(ss,[21,21],9,[0.1 30.0; -0.1
20.0],1e-8,1e-6,1000)}

\subsection{Model solution}

Once a SolutionScheme is specified we are in a position to solve the model.
\ In order to do so we use the solve\_model() function, which takes either
two or three arguments. \ For a perturbation solution solve\_model()
requires two arguments: the model to be solved and the SolutionScheme, as
follows:

\bigskip

\textit{soln\_first\_order = solve\_model(dsge,N)}

\bigskip

\textit{soln\_second\_order = solve\_model(dsge,NN)}

\bigskip

\textit{soln\_third\_order = solve\_model(dsge,NNN)}

\bigskip

Alternatively, for a projection solution solve\_model() takes either two or
three arguments. \ To provide a concrete example, suppose we wish to solve
our model using Chebyshev polynomials. \ If we want the projection solution
to be initialized using the steady state, then solve\_model() requires only
two arguments: the model to be solved and the SolutionScheme:

\bigskip

\textit{soln\_chebyshev = solve\_model(dsge,C)}

\bigskip

If we want the projection solution to be initialized using the third order
perturbation solution, then solve\_model() requires three arguments: the
model to be solved, the initializing solution, and the SolutionScheme:

\bigskip

\textit{soln\_chebyshev = solve\_model(dsge,soln\_third\_order,C)}

\bigskip

Although this example uses a third order perturbation as the initializing
solution, any solution (first order, second order, third order, Chebyshev,
Smolyak, or piecewise linear) can be used.

\subsubsection{A comment on third-order perturbation}

Sometimes it can be useful to add skewness to the shocks, but this is not
easy to do through the model file. \ If you want your shocks to be skewed,
then you can access the third order perturbation solution by calling

\bigskip

\textit{soln\_third\_order = solve\_third\_order(dsge,NNN,skewness)}

\bigskip

where skewness is a 2D array containing the skewness coefficients. \ If
there is only one shock, then the skewness array is%
\begin{equation*}
skewness=E\left[ \epsilon _{1}\epsilon _{1}\epsilon _{1}\right] .
\end{equation*}%
If there are two shocks, then the skewness array is%
\begin{equation*}
skewness=E\left[ 
\begin{array}{cccc}
\epsilon _{1}\epsilon _{1}\epsilon _{1} & \epsilon _{1}\epsilon _{1}\epsilon
_{2} & \epsilon _{1}\epsilon _{2}\epsilon _{1} & \epsilon _{1}\epsilon
_{2}\epsilon _{2} \\ 
\epsilon _{2}\epsilon _{1}\epsilon _{1} & \epsilon _{2}\epsilon _{1}\epsilon
_{2} & \epsilon _{2}\epsilon _{2}\epsilon _{1} & \epsilon _{2}\epsilon
_{2}\epsilon _{2}%
\end{array}%
\right] .
\end{equation*}%
Etc.

\subsubsection{Solution structures}

When a model is solved the solution is returned in the form of a structure.
\ The exact structure returned depends on the solution method.

\paragraph{First-order perturbation}

The first-order perturbation solution takes the following form:%
\begin{eqnarray*}
\mathbf{x}_{t+1} &=&\mathbf{h}_{\mathbf{x}}\mathbf{x}_{t}+\mathbf{k\epsilon }%
_{t+1}, \\
\mathbf{y}_{t} &=&\mathbf{g}_{\mathbf{x}}\mathbf{x}_{t}.
\end{eqnarray*}%
The solution structure for a stochastic first-order perturbation has the
following fields:

\begin{itemize}
\item hbar --- The steady state of the state variables

\item hx --- The first-order coefficients in the state-transition equation

\item k --- The loading matrix on the shocks in the state-transition
equation.

\item gbar --- The steady state of the jump variables

\item gx --- The first-order coefficients in the jump's equation

\item sigma --- An identy matrix

\item grc --- The number of eigenvalues with modulus greater than cutoff.

\item Soln\_type --- Either \textquotedblleft determinate\textquotedblright
, \textquotedblleft indeterminate\textquotedblright , or \textquotedblleft
unstable\textquotedblright .
\end{itemize}

The solution to a deterministic model has the same fields as the stochastic
solution with the exceptions of $\mathbf{k}$ and sigma.

\paragraph{Second-order perturbation}

The second-order perturbation solution takes the following form:%
\begin{eqnarray*}
\mathbf{x}_{t+1} &=&\mathbf{h}_{\mathbf{x}}\mathbf{x}_{t}+\frac{1}{2}\mathbf{%
h}_{\mathbf{ss}}+\frac{1}{2}\left( \mathbf{I}\otimes \mathbf{x}_{t}\right) 
\mathbf{h}_{\mathbf{xx}}\left( \mathbf{I}\otimes \mathbf{x}_{t}\right) +%
\mathbf{k\epsilon }_{t+1}, \\
\mathbf{y}_{t} &=&\mathbf{g}_{\mathbf{x}}\mathbf{x}_{t}+\frac{1}{2}\mathbf{g}%
_{\mathbf{ss}}+\frac{1}{2}\left( \mathbf{I}\otimes \mathbf{x}_{t}\right) 
\mathbf{g}_{\mathbf{xx}}\left( \mathbf{I}\otimes \mathbf{x}_{t}\right) .
\end{eqnarray*}

The solution structure for a stochastic second-order perturbation has the
following fields:

\begin{itemize}
\item hbar --- The steady state of the state variables

\item hx --- The first-order coefficients in the state-transition equation

\item hss --- The second-order stochastic adjustment to the mean in the
state-transion equation

\item hxx --- The second-order coefficients in the state-transition equation

\item k --- The loading matrix on the shocks in the state-transition
equation.

\item gbar --- The steady state of the jump variables

\item gx --- The first-order coefficients in the jump's equation

\item gss --- The second-order stochastic adjustment to the mean in the
jump's equation

\item gxx --- The second-order coefficients in the jump's equation

\item sigma --- An identy matrix

\item grc --- The number of eigenvalues with modulus greater than cutoff.

\item Soln\_type --- Either \textquotedblleft determinate\textquotedblright
, \textquotedblleft indeterminate\textquotedblright , or \textquotedblleft
unstable\textquotedblright .
\end{itemize}

The solution to a deterministic model has the same fields as the stochastic
solution with the exceptions of $\mathbf{h}_{\mathbf{ss}}$, $\mathbf{k}$, $%
\mathbf{g}_{\mathbf{ss}}$, and sigma.

\paragraph{Third-order perturbation}

The third-order perturbation solution takes the following form:%
\begin{eqnarray*}
\mathbf{x}_{t+1} &=&\mathbf{h}_{\mathbf{x}}\mathbf{x}_{t}+\frac{1}{2}\mathbf{%
h}_{\mathbf{ss}}+\frac{1}{2}\mathbf{h}_{\mathbf{xx}}\left( \mathbf{x}%
_{t}\otimes \mathbf{x}_{t}\right) +\frac{1}{6}\mathbf{h}_{\mathbf{sss}}+%
\frac{1}{6}\mathbf{h}_{\mathbf{ssx}}\mathbf{x}_{t}+\frac{1}{6}\mathbf{h}_{%
\mathbf{xxx}}\left( \mathbf{x}_{t}\otimes \mathbf{x}_{t}\otimes \mathbf{x}%
_{t}\right) +\mathbf{k\epsilon }_{t+1}, \\
\mathbf{y}_{t} &=&\mathbf{g}_{\mathbf{x}}\mathbf{x}_{t}+\frac{1}{2}\mathbf{g}%
_{\mathbf{ss}}+\frac{1}{2}\mathbf{g}_{\mathbf{xx}}\left( \mathbf{x}%
_{t}\otimes \mathbf{x}_{t}\right) +\frac{1}{6}\mathbf{g}_{\mathbf{sss}}+%
\frac{1}{6}\mathbf{g}_{\mathbf{ssx}}\mathbf{x}_{t}+\frac{1}{6}\mathbf{g}_{%
\mathbf{xxx}}\left( \mathbf{x}_{t}\otimes \mathbf{x}_{t}\otimes \mathbf{x}%
_{t}\right) .
\end{eqnarray*}

The solution structure for a stochastic third-order perturbation has the
following fields:

\begin{itemize}
\item hbar --- The steady state of the state variables

\item hx --- The first-order coefficients in the state-transition equation

\item hss --- The second-order stochastic adjustment to the mean in the
state-transion equation

\item hxx --- The second-order coefficients in the state-transition equation

\item hsss --- The third-order stochastic adjustment to the mean in the
state-transition equation

\item hssx --- The skewness adjustment \ othe state-transition equation

\item hxxx --- The third-order coefficents in the state-transition equation

\item k --- The loading matrix on the shocks in the state-transition
equation.

\item gbar --- The steady state of the jump variables

\item gx --- The first-order coefficients in the jump's equation

\item gss --- The second-order stochastic adjustment \ othe mean in the
jump's equation

\item gxx --- The second-order coefficients in the jump's equation

\item gsss --- The third-order stochastic adjustment to the mean in the
jump's equation

\item gssx --- The skewness adjustment in the jump's equation

\item gxxx --- The third-order coefficients in the jump's equation

\item sigma --- An identy matrix

\item grc --- The number of eigenvalues with modulus greater than cutoff.

\item Soln\_type --- Either \textquotedblleft determinate\textquotedblright
, \textquotedblleft indeterminate\textquotedblright , or \textquotedblleft
unstable\textquotedblright .
\end{itemize}

The solution to a deterministic model has the same fields as the stochastic
solution with the exceptions of $\mathbf{h}_{\mathbf{ss}}$, $\mathbf{h}_{%
\mathbf{sss}}$, $\mathbf{h}_{\mathbf{ssx}}$, $\mathbf{k}$, $\mathbf{g}_{%
\mathbf{ss}}$, $\mathbf{g}_{\mathbf{sss}}$, $\mathbf{g}_{\mathbf{ssx}}$, and
sigma.

\paragraph{Chebyshev solution}

The solution structure for the Chebyshev solution has the following fields:

\begin{itemize}
\item variables \ --- A vector of arrays containing the solution for each
variable

\item weights --- A vector of arrays containing the weights for the
Chebyshev polynomials

\item nodes --- A vector of vectors containing the Chebyshev nodes

\item order --- The order of the Chebyshev polynomials

\item domain --- The domain for the state variables

\item sigma --- The variance-covariance matrix for the shocks

\item iteration\_count --- The number of iterations needed to achieve
convergence
\end{itemize}

The solution to a deterministic model has the same fields with the exception
of sigma.

\paragraph{Smolyak solution}

The solution structure for the Smolyak solution has the following fields:

\begin{itemize}
\item variables --- A vector of arrays containing the solution for each
variable

\item weights: --- A vector of vectors containing the weights for the
Chebyshev polynomials

\item grid --- A matrix containing the Smolyak grid

\item multi\_index --- A matrix containing the multi-index underlying the
polynominals

\item layer --- The number of layers in the approximation

\item domain --- The domain for the state variables

\item sigma --- The variance-covariance matrix for the shocks

\item iteration\_count --- The number of iterations needed to achieve
convergence
\end{itemize}

The solution to a deterministic model has the same fields with the exception
of sigma.

\paragraph{Piecewise linear solution}

The solution structure for the piecewise linear solution has the following
fields:

\begin{itemize}
\item variables --- A vector of arrays containing the solution for each
variable

\item nodes --- A vector of vectors containing the Chebyshev nodes

\item domain --- The domain for the state variables

\item sigma --- The variance-covariance matrix for the shocks

\item iteration\_count --- The number of iterations needed to achieve
convergence
\end{itemize}

The solution to a deterministic model has the same fildls with the exception
of sigma.

\section{Post-solution analysis}

Once you have solved your model there are many things that you might want to
use the solution for. \ Some of the more obvious things, such as simulating
data from the solution and computing impulse response functions have been
built into SolveDSGE to make things easier for you.

\subsection{Simulation}

To simulate data from the solution to a model the function to use is
simulate(), whose arguments are a model solution, an initial state, and the
number of observations to simulate. \ An optimal final argument is the seed
for the random number generator. \ An example of simulate() in action might
be

\bigskip

\textit{data\_states, data\_jumps = simulate(soln,[0.0, 25.0],100000)}

\bigskip

As this example makes clear, the simulate function returns two 2D arrays. \
The first array contains simulated data for the state variables, the second
array contains simulated data for the jump variables. \ The simulate
function can be applied to both stochastic and deterministic models.

\subsection{Impulse response functions}

Impulse responses are obtained using the impulses() function, which takes
three arguments: the model solution, the length of the impulse response
function (number of periods), the number of the shock to apply the impulse
to, and the number of repetitions to use for the Monte Carlo integration. \
Responses to both a positive and a negative innovation are generated. \ An
optimal final argument is the seed for the random number generator. \ An
example of impulses() in use might be

\bigskip

pos\_\textit{responses, neg\_responses = impulses(soln,50,1,10000)}

\bigskip

For the nonlinear solutions (second-order perturbation, third-order
perturbation, and the projection-based solutions) the initial state is
\textquotedblleft integrated-out\textquotedblright\ via a Monte Carlo that
averages over draws taken from the unconditional distribution of the state
variables. At this stage in the package's development, the impulses need to
be computed one shock at a time; this will probably change at some point.

\subsection{PDFs and CDFs}

SolveDSGE contains functions for approximating the probability density
function and the cumulative distribution function of a variable, where the
approximation is based on Fourier series (Kronmal and Tarter, 1968). \ Most
of the functionality relates to the univariate case, but some functionality
is included to approximate and evaluate the PDF\ in the multivariate case.

\subsubsection{Univariate}

To approximate the probability density function and evaluate the
approximated function at a point the function is

\bigskip

\textit{f =
approximate\_density(sample,point,order,lower\_bound,upper\_bound)}

where \textit{sample} is a vector of data, \textit{point} is the value at
which the PDF is evaluated, \textit{order} is the order of the Fourier
series approximation, and \textit{lower\_bound} and \textit{upper\_bound}
specify the support over which the PDF is constructed. \ If an approximation
of the entire PDF is sought, then the function is

\bigskip

\textit{nodesf, f =
approximate\_density(sample,order,lower\_bound,upper\_bound)}

\bigskip

Similarly, the cumulative distribution function is approximated and
evaluated at a point using the function

\bigskip

\textit{F =
approximate\_distribution(sample,point,order,lower\_bound,upper\_bound)}

\bigskip

where \textit{sample} is a vector of data, \textit{point} is the value at
which the CDF is evaluated, \textit{order} is the order of the Fourier
series approximation, and \textit{lower\_bound} and \textit{upper\_bound}
specify the support over which the CDF is constructed. \ If an approximation
of the entire CDF is sought, then the function is

\bigskip

\textit{nodesF, F =
approximate\_distribution(sample,order,lower\_bound,upper\_bound)}

\bigskip

\subsubsection{Multivariate}

In the multivaraite case the PDF can be approximated and evaluated at a
point using the function

\bigskip

\textit{f =
approximate\_density(sample,point,order,lower\_bound,upper\_bound)}

\bigskip

where \textit{sample} is a 2D array of data, \textit{point} is a vector, 
\textit{order} is a vector of integers, and \textit{lower\_bound} and 
\textit{upper\_bound} are vectors. \ From this function, marginal densities
and conditional densities can be computed. \ Similarly, the CDF can be
approximated and evaluated at a point using the function

\textit{F =
approximate\_distribution(sample,point,order,lower\_bound,upper\_bound)}

\bigskip

where \textit{sample} is a 2D array of data, \textit{point} is a vector, 
\textit{order} is a vector of integers, and \textit{lower\_bound} and 
\textit{upper\_bound} are vectors.

\begin{thebibliography}{99}
\bibitem{} Andreasen, M., Fern\'{a}ndez-Villaverde, J., and J.
Rubio-Ramirez, (2017), \textquotedblleft The Pruned State-Space System for
Non-Linear DSGE Models: Theory and Empirical Applications\textquotedblright
, \textit{Review of Economic Studies}, 0, pp. 1---49.

\bibitem{} Binning, A., (2013), \textquotedblleft Third-order approximation
of dynamic models without the use of tensors\textquotedblright , \textit{%
Norges Bank Working Paper} 2013--13.

\bibitem{} Gomme, P., and P. Klein, (2011), \textquotedblleft Second-Order
Approximation of Dynamic Models Without the Use of Tensors\textquotedblright
, \textit{Journal of Economic Dynamics and Control}, 35, pp. 604---615.

\bibitem{} Judd, K. (1992), \textquotedblleft Projection Methods for Solving
Aggregate Growth Models\textquotedblright , \textit{Journal of Economic
Theory}, 58, pp.410---452.

\bibitem{} Judd, K., Maliar, L., Maliar, S., and R. Valero, (2014),
\textquotedblleft Smolyak Method for Solving Dynamic Economic Models:
Lagrange Interpolation, Anisotropic Grid and Adaptive
Domain\textquotedblright , \textit{Journal of Economic Dynamics and Control}%
, 44, pp. 92---123.

\bibitem{} Judd, K., Maliar, L., Maliar, S., and I. Tsener, (2017),
\textquotedblleft How to Solve Dynamic Stochastic Models Computing
Expectations just Once\textquotedblright , \textit{Quantitative Economics},
8, pp.851---893.

\bibitem{} Klein, P., (2000), \textquotedblleft Using the Generalized Schur
Form to Solve a Multivariate Linear Rational Expectations
Model\textquotedblright , \textit{Journal of Economic Dynamics and Control},
24, pp. 1405---1423.

\bibitem{} Kronmal, R., and M. Tarter, (1968), \textquotedblleft The
Estimation of Probability Densities and Cumulatives by Fourier Series
Methods\textquotedblright , \textit{Journal of the American Statistical
Association}, 63, 323, pp.925--952.

\bibitem{} Levintal, O., (2017), \textquotedblleft Fifth-Order Perturbation
Solution to DSGE models\textquotedblright , \textit{Journal of Economic
Dynamics and Control}, 80, pp. 1--16.

\bibitem{} Potter, S., (2000), \textquotedblleft Nonlinear Impulse Response
Functions\textquotedblright , \textit{Journal of Economic Dynamics and
Control}, 24, pp. 1425---1446.
\end{thebibliography}

\end{document}
